\begin{frame}{Configuration}
    The first time you use GitHub on the command line,
    you need to configure your username and email.

    \texttt{git config --global user.name ``your\_username''}

    \texttt{git config --global user.name youremail@example.com}

    GitHub uses personal access tokens (PATs) instead of passwords.

    Follow the steps outlined \href{https://docs.github.com/en/authentication/keeping-your-account-and-data-secure/creating-a-personal-access-token}{here} to get a PAT to use
    when prompted for your password.
\end{frame}

\begin{frame}{Cloning a repository}
    Go to the repository, click on the big green `Code` button, and copy the link.

    Open the command line, navigate to the parent folder that you the project
    to exist in, and type:

    \texttt{git clone [link]}

    It should download the project into a folder with the repository name.
\end{frame}

\begin{frame}{Commits and branches}
    When in a GitHub repository, you live at a certain \texttt{commit} on a certain branch.

    Commits are nodes that correspond a set of changes to code.

    If you only commit to the \texttt{main} (default) branch,
    your project history is just a series of changes from the first commit to your
    most recent commit.
\end{frame}

\begin{frame}{Commits and branches}
    \stretchon
    You can create also a new \texttt{branch},
    which will become its own series new commits that are not incorporated into the
    main code.

    When you decide to incorporate the new branch into the \texttt{main}
    (on GitHub this normally means that a teammate has reviewed and approved
    the changes via a pull request),
    you can \texttt{merge} the changes from the branch into \texttt{main}.
\end{frame}


\begin{frame}{The basic toolkit}
    \stretchon
    \begin{itemize}
        \item[]\texttt{git pull}: ``Grab and apply the changes that I don't have''
        \item[]\texttt{git branch [branch name]}\footnote{
                \scriptsize Branching off of the commit that you are currently located at.
            }: ``Create a branch''
        \item[]\texttt{git branch}: ``Show me branches that I could switch to''
        \item[]\texttt{git switch [branch name]}: ``Switch to a certain branch''
        \item[]\texttt{git status}: ``Show me the state of directory''
        \item[]\texttt{git restore [files]}: ``Undo changes that I have made''
        \item[]\texttt{git clean}: ``Undo changes that I have made''
        \item[]\texttt{git add [files]}: ``Stage these changes to be committed''
        \item[]\texttt{git commit -m ``commit message''}: ``Commit staged changes''
        \item[]\texttt{git push}: ``Send commits that I have made off to GitHub''
    \end{itemize}
\end{frame}

\begin{frame}{Git checkout and other commands to check out}
    \stretchon
    \begin{itemize}
    \item[]\texttt{git checkout [branch name]}:

        ``Switch to the branch''
    \item[]\texttt{git checkout [branch name] -- [file]}:

        ``Restore the file to the state it
    is at in the specified branch''
    \end{itemize}

    If you are spending a lot of time integrating changes from one branch into
    another, look into \texttt{rebase} and \texttt{cherry-pick}.
\end{frame}

\begin{frame}{Worktrees}
    \stretchon
    I just discovered the \texttt{git worktree} command,
    and it is great.

    If you work on multiple branches simultaneously often and don't want to constantly
    be switching between them, start using worktrees.
\end{frame}

\begin{frame}{git diff}
\end{frame}

\begin{frame}{git bisect}
\end{frame}

\begin{frame}{git add --patch}
\end{frame}

\begin{frame}{GitHub CLI}
    \stretchon
    Also look into \href{https://cli.github.com/}{GitHub CLI}.

    It allows you to do most of what you can do on the website from the terminal.

    \textbf{Example:} \texttt{gh repo create new\_repo}

    \textbf{Example:} \texttt{gh issue list -a ljmotley}

    \textbf{Example:} \texttt{gh browse}

\end{frame}
