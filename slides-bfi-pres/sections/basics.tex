
\begin{frame}{What is the command line?}
    \textbf{The interface that allows you to perform actions on your computer with commands.}

    \begin{itemize}
        \item Basically synonyms\footnote{I suspect this is pragmatically true for economists.
        Personally, I am not good enough to need to know the difference between them.
    I open my \textit{terminal}/\textit{command line}/\textit{shell} and do something.}:
    \textit{terminal}, \textit{shell}, \textit{command line}, \textit{CLI}, \textit{command prompt}.
    \end{itemize}
\end{frame}

\begin{frame}{Disclaimer}
    \footnotesize
    I am on a Mac and use \textit{zsh} shell.

    The slides will refer to \textit{bash} shell, since it is more common.

    The basic syntax identical, and the commands themselves are almost the same
    (disclaimer: there are headache-inducing edge cases).

    If you're on Windows, you will not have a \textit{bash} shell by default,
    but you can install one.

    I'll also use this slide to mention that I have only recently come to use
    all of this stuff into my workflow.
    I am definitely not qualified to teach this.
    But these tools have totally changed the way to approach the job,
    so hopefully I can at least convince you to give them a try.

\end{frame}

\begin{frame}{Command to English}
    \stretchon
    At the beginning, you will forget a lot of these commands and have to Google
    how to do things.

    A silly but helpful trick is to deliberately think what the command is doing in English
    when you use the command.

    Eventually, you can just think in English while you work and habit gets your
    fingers to do the right thing.\footnote{This is also (and especially) true
    for keyboard shortcuts, which are an important tool if you use a terminal-based
    text editor.}

\end{frame}

\begin{frame}{Basic syntax}
    \stretchon
    \textbf{Syntax:} \texttt{[command]} \texttt{[option(s)]} \texttt{[argument(s)]}

    \textbf{Example:} \texttt{find output -name `robustnesscheck87.png'}

    Sometimes commands have implied arguments when unspecified:

    \textbf{Example:} \texttt{cd} (implied $\sim$)
\end{frame}

\begin{frame}{Getting help}
    \stretchon
    Ask for help on a command with \texttt{-h} or \texttt{-help}.

    \textbf{Example:} \texttt{git commit -h}

    When \texttt{-h} doesn't work, there often is a manual entry that can be
    accessed with \texttt{man [command]}.

    \textbf{Example:} \texttt{man cat}

    Bonus (on macOS or Linux): install \href{https://github.com/chrisallenlane/cheat}{cheat}.
\end{frame}

\begin{frame}{Moving around}
    \stretchon
    \begin{itemize}
        \item[\texttt{pwd}:] ``Where am I?''
        \item[\texttt{cd}:] ``Move to''
        \item[\texttt{ls}:] ``List''
        \item[\texttt{.} :] ``Here''
        \item[\texttt{..} :] ``Backwards one directory''
        \item[\texttt{$\sim$}:] ``User directory''
        \item[\texttt{\textbackslash}:] ``Base directory''
    \end{itemize}

    \textbf{Example:} \texttt{cd ../..} ``Move backwards two directories''

    \textbf{Example:} \texttt{ls input} ``List files in the input folder''
\end{frame}

\begin{frame}{Moving around}
    \textbf{Warning:} \texttt{\textbackslash} tells \textit{bash} to interpret the next character
    `literally'.
    This is required with spaces.

    If you named a folder \texttt{A Folder}, you would need to \texttt{cd A}\textbackslash\hspace{0.1cm}\texttt{Folder}.
    \footnote{Another benefit of using the command line is that it encourages you to name your folder
    \texttt{my\_folder} instead.}
\end{frame}

\begin{frame}{Moving around}
    \begin{itemize}
        \item[\texttt{find}:] ``Find (files or folders)''
    \end{itemize}



    \textbf{Example:} \texttt{find a\_folder -iname `*.txt'}

    ``Find text files in \texttt{a\_folder}''

    \textbf{Example:} \texttt{find . -type d -iname `a\_folder'}

    ``Find folders with name \texttt{a\_folder}''

    \textbf{Example:} \texttt{find . -size +100k}

    ``Find files that are larger than 100kb''
\end{frame}

\begin{frame}{Creating and destroying things}
    \stretchon
    \begin{itemize}
        \item[\texttt{mkdir}:] ``Make a directory''
        \item[\texttt{touch}:] ``Create a file''
        \item[\texttt{rm}:] ``Destroy a file''
        \item[\texttt{rmdir}:] ``Destroy a directory'' (fails if not empty)
        \item[\texttt{rm -r}:] ``Destroy this and everything it contains''
    \end{itemize}

    \textbf{Example:} \texttt{touch hello.txt}
    ``Create an empty .txt file called \texttt{hello.txt}.''

    \large \textbf{Warning:} \texttt{rm} \textbf{really does destroy,
    it doesn't move to Trash}

\end{frame}

\begin{frame}{Other common operations}
    \stretchon
    \begin{itemize}
        \item[\texttt{cp}:] ``Copy this''
        \item[\texttt{cp -r}:] ``Copy this and everything it contains''
        \item[\texttt{mv}:] ``Move (or rename, if moving to same directory)''
    \end{itemize}
    \vspace{0.4cm}

    \textbf{Example:} \texttt{mv hello.txt goodbye.txt} ``rename \texttt{hello.txt} to \texttt{goodbye.txt}''
\end{frame}

\begin{frame}{Wildcards}
    If you've stopped paying attention and don't know about these, pay attention again.

    \begin{itemize}
        \item[\texttt{*}:] ``I can be anything''
        \item[\texttt{?}:] ``I can be one thing''
    \end{itemize}


    \textbf{Example:} \texttt{mv input/*.txt .} ``move all text files in the input directory here''

    \textbf{Example:} \texttt{ls hello\_?.txt} ``lists \texttt{hello\_1.txt} but not \texttt{hello\_10.txt}''
\end{frame}

\begin{frame}{Echo and cat}
    \begin{itemize}
        \item[\texttt{echo}] ``Print''
        \item[\texttt{cat}] ``Print the contents of a file''
        \item[\texttt{>}] ``write to file (overwrites if it exists)''
        \item[\texttt{>>}] ``append to file''
    \end{itemize}

\textbf{Example:} \texttt{echo `Hello' > hello.txt}

``Create text file with `Hello' on the first line''

\textbf{Example:} \texttt{cat hello.txt > goodbye.txt}

``Copies everything from \texttt{hello.txt} to \texttt{goodbye.txt}''

\textbf{Example:} \texttt{echo `Goodbye' >> goodbye.txt}

``Adds `Goodbye' to the last line of \texttt{goodbye.txt}''

\end{frame}

\begin{frame}{Head and tail}
    \begin{itemize}
        \item[\texttt{head} ] ``Show me the beginning of the file''
        \item[\texttt{tail}\footnote{
        \textit{Note:} \texttt{tail} is useful for debugging.
        We often do \texttt{tail stata\_do\_file\_that\_failed.log}
    to quickly see the final error message.}]
        ``Show me the end of the file''
    \end{itemize}

    \textbf{Example:} \texttt{head `goodbye.txt'}

    ``Show me the first 10 line of \texttt{goodbye.txt} ''

    \textbf{Example:} \texttt{tail -n 3 `hello.txt'}

    ``Show me the last 3 lines of \texttt{hello.txt} ''

\end{frame}

\begin{frame}{Sed, grep, and piping}
    \texttt{sed}
    \footnote{I've seen people on Stack Overflow advocate for \texttt{awk} over \texttt{sed}
    on a case-by-case basis
    (and it may be especially better for \texttt{.csv} editing),
    but I have less experience with \texttt{awk}.}
    and \texttt{grep} are great and can do almost any task in their domain
    (though you may need to Google around some to find the right flag).


    \begin{itemize}
        \item[\texttt{sed}:] ``Substitute''
        \item[\texttt{grep}:] ``Find (text)''
        \item[\texttt{|}:] ``Pass to the next command''
    \end{itemize}

    \textbf{Example:} \texttt{grep -Rl ``hello''}

    \textbf{Example:} \texttt{sed -i ``s/hello/goodbye/'' hello.txt}

    \textbf{Example:} \scriptsize \texttt{grep -o ``input/[A-Za-z0-9]*.[a-z]'' | xargs sed -i ``s/input/output/g''}

\end{frame}

\begin{frame}{Aliases}
    \stretchon
    Aliases are self-defined shortcuts.

    You define these in a script file that executes every time you launch
    your terminal.

    The file is in your user directory and is called \texttt{.bashrc} (or \texttt{.zshrc}).

    \textbf{Syntax:} \texttt{alias="[full command]"}


    This is also where your \texttt{PATH} lives (I can add notes on this).
\end{frame}
